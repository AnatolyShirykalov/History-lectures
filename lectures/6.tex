
Итак, причины возвышения Московского княжества. Есть причины объективные, есть субъективны. Объективные причины
\begin{itemize}
\item Выгодность положения. В бывшей Киевской Руси Москва на окрание, но в центре формировавшейся Руси, Великоруссии. А Киев как раз на окраине обоих формирований. Именно поэтому Москва, пока не была в столице, не запустела. Нынче Петербургу помогают, обязывая предприятия переезжать, искусственные меры. Москву никто не поддерживал, а она состоялась. Московские железные дороги построили при царе. Железные дороге уже тогда на Москве сошлись. Положение центральное.

\item Есть и некоторая защищённость от Орды, от Литвы. Но не будем преувеличивать, от Твери не была уже защищена.
\item Река Москва "--- важнейший приток Оки. Это часть великого Волжского торгового пути.
\end{itemize}

Главное занятие земледелие в те времена. А земли в Москва неплодородны. Между Суздалем и Ростовым место без леса и в те времена и сейчас почему-то. Есть такое древнее название древнего города Переяславль залесский. Лес кончается, потом снова уже лес. 

Итак, земли малоплодородны, то есть не всё хорошо. Техники везде была примитивной, всё решала плодовитость земли. Поэтому древнейшие центры цивилизации "--- это субтропики: жарко и вода есть.

Теперь перейдём к фактору субъективному: личности Московских князей. Они потоми Мономаха, потомки знаменитейшего Александра Невского. Самый известный и выдающийся "--- это Иван Калита.

Этих князей недооценили. Их историки называют собирателями русских земель. Без пафоса писали древних князей. Средние люди, как средний спортсмен, средний профессионал. Но это был фактор положительный. То, что они не считали себя героями, они действовали расчётливо, наверняка. Не были воинственными, хотя эпоха неспокойная, в Европе как раз все воинственные. А князья были дипломатами. Было упорство: из поколения в поколение движение к одной цели.

У нас есть недостаток: стремление к авторитарности власти. История по царям. И сейчас эпохи Ленина, Сталина, Горбачёва, Ельцина, Путина. А вот у Англии нет вечных союзников, есть вечные интересы. И они задаются институтами. У нас политика сильно зависит от характера текущего монарха.

Цель князя тех времён: сделать княжество чуть больше и передать сыновьям. Постепенно незаметно эта материальная цель приобретает высокое значение: идёт собирание земель, преодоление раздробленности. Итак, князья были не мечтатели, не рискуют (хотя воюют, когда надо воевать, но не любят), проводят простую политику: то что есть приумножить.

Калита заложил основу Московского могущества. А он ещё был послушным Ордынскому Хану. Орда не мешала возвышению Москвы, потому что считали Московских князей своими сателитами, агентами. И постоянно шли восстания и передали сбор дани князьям. Отвоз дани в одру был поручен уже Калите.

А вот Донской уже вступил в бой с Монголами. Почему? Калита трус, а Донской храбрец? Соотношение сил стало другим. При Калите Орда была в расцвете, а при Донском там двадцать лет была смута.

Процесс завершается на рубеже 15--16 веков образованием государства. Это Иван Третий, Василий Третий. Ещё одна недооценённая фигура, сопоставимая с Петром Первым. Итак при Иване Третьем (его современники называли Великим, и заслужил). При Иване Третьем Новгород вошёл в состав Руси, а это громадная территория, Тверь вошла. Василий подчиняет Ростов, Псков, но они уже при Иване Третьем были зависимы от Москвы. Но все ли Земли были собраны? Русскими землями тогда называли земли православные (нынешняя Беларусия, часть Украины), территория бывшей Киевской Руси.

Василий захватывает у Литовцев Смоленск. Границы с Литвой в начале это истории были где-то за Можайском. Территория древней Руси были либо у Москвы, либо у Литвы. Независимых объединений больше не было.

Итак, второй фактор независимость. Зависимость от монголов пала.

Третий фактор. Появилась новая символика, новая культура, Кремль строят. Новый режим всегда строит свою архитектуру. Пётр Первый строит Петербург как символ новой Империи, Сталин строит высотную Москву. Все великие храмы, как Исакиевский собор, "--- это гос. заказы. Никакому богачу это не по карману. Сейчас вообще нет интересных явлений искусства. И это неслучайно. Когда появляется новое государство, новый строй "--- это воодушевляет.

Иван Четвёртый захотел повысить свой титул. Тогда была иерархия титулов: князи, герцоги. Повыше сложно. Почему? Более высокий титул это корона. Король "--- это искажённое имя Карла Великого (он стал императором вообще-то), которое стало титулом. Итак, попытка получить царский титул. А как произошёл этот титул? Это понятие происходит от Caesar "--- это тоже личное имя. В Риме стеснялись себя называть Цезарями, но давали имена. Это титулы Римской Империи. Получается, титул "--- это император. Отсюда и Германское название императора: Кайзер. Царьградом называли Константинополь "--- а это столица империи.

Когда при Петре появляется титул Император, не пропадает полностью слово Царь. Остаётся термин «Царская Россия», например. При Иване впервые проводят процедуру коронования. Зачем это всё нужно? Должно сообщить власти Монарха больший престиж. Чтобы показать, что по воли бога. Не простой человек, а особый: вот символ. В брак вступают Монархи только между собой: только с представителями суверенных династий. Чтобы никто не думал: «а чего он сидит на престоле? По какому праву?». Чтобы никто не посягал на престол.

Иван Третий решил провести процедуру коронации в пользу внука. Но потом пепедумал. Потом власть перешла к Василию. Почему так жестоко с внуком поступил? Чтобы положить конец конкуренции. В Московском княжестве была усобица. Кто соперничал: дядя Юрий Звенигородский и его племянник Василий второй. Не могли понять, кто должен унаследовать престол Василия первого. Ради государства нужно пожерствовать или сыном или внуком. Те, кто своих сынков во власть тянет, не может претендовать на звание Великий. Это для банальных людей. А вот Василий выше этого.

Появляется новый Герб России. Раньше гербом был тот, что сейчас герб Москвы. А теперь появляется двуглавый орёл.

\section{Царская Россия}
Далее начинается двухвековая история царства. Чем отличается гос строй Московского государства от древнерусского? Он стал авторитарнее. Именно в это время эта традиция и возникает. Князь на месте был как судья. Судить по праву, по закону. Судья своим подсуживает, но у князя-то нет родства с подданными, он может судить объективно. Проводить внешнюю политику проще, потому что приходится всё время общаться с монархами. В Европе в конце концов государи друг другу письма начинали со слов «Государь, брат мой». Они были браться. Если «брат» не писать, то можно обидеть. Это стандартная фраза была.

В среднем князь в Новгороде проводил 4--5 лет. Потом его меняли. 

Почему возникает в Москве авторитарная традиция. Несколько факторов.
\begin{itemize}
\item Последствие могнольского нашествия. Здесь две причины. Во-первых, монгольское нашествие, как считается, сильно повлияло на нравственность. А грубое общество имеет и грубую власть. И наоборот, грубая власть даёт грубый пример обществу. Позже Пушкин скажет (к современной России не подходит) «В России правительство "--- единственный европеец». Второй фактор: могнолы впервые познакомили с властью, которой нужно подчиняться бесприкословно. Власть тираническая. Это общение стало примером. Падение зависимости от Орды означало, что ханская ставка была перенесена ву Москву. Московские цари перенесли эту традицию власти  в Москву. Это скорее публицистика, но доля правды есть. Третий фактор: в ходе нашествия больше всего города пострадали (рыскать по лесам, а чего там найдёшь? пару медяков); города "--- все богатства: казна, епископ, церковное богаство. Сильно боярство пострадало; если осада, то они сражаются. Почти всё Московское боярство не помнит свою родословную до монгольского нашествия. И в дальнейшей истории роль города сильно упала. Нет буржуазии.
\item Возникшее и растущее Московское княжество оказалось зажато между двумя противниками: орда на востоке, Литва на западе. Позже и на юге Турция. Чтобы вызвать конкуренцию, расшириться и победить в этой борьбе, нужна концентрация немногих ресурсов и власти в одних руках. Позже наши монархисты девятнадцатого века: «самодержавия "--- строй, который вырабатон нашей историе». Самодержавие себя оправдала. Ничтожное княжество превратилось в самою обширную империю в истории. И оно далеко не исчерпало себя.
\end{itemize}

Мы, студенты, преподаватели, не любим авторитаризм. У нас излишне сильная власть. Это банально. Но это не значит, что это было плохо тогда. Многие страны загибались от споров, от конституции. Для каждой эпохи свои реальности.
