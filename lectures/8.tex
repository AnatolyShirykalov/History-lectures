
Мы остановились на общем смысле его политики. О том, что он создаёт в России режим абсолютизма: патриарх исчезает, боярская дума исчезает, земский собор сам исчезает. Ввёл завещательный порядок передачи престола. Последнне "--- это то, что Пётр первый был абсолютным монархом не только фактически, но и формально. Те, кто правили до Петра первого имели ту же власть, тот же титул. Являлись ли они действительно монархами? Нет.

\subsection{Преобразование системы гос. управления}
Какие идеи исповедовал Пётр первый. Он не был теоретиком. Он был практиком, эмпириком. Подбирал то, что ему соответствовало. Он не был догматиком, который увидел идею и потом её преследует. Он гордился тем, что сам может спроектировать и построить корабль. Знает все сопутствующие ремёсла. Тем не менее некоторые идеи раздеял. Например, государство общего блага. 
Далее с этой идеей был связана идея общественного договора. Что государство возникло как общественный договор, чтобы был порядок. Люди пожертвовали своей первобытной вольностью, чтобы государство преследовало благо общее. Не нужно этой идеей любоваться. Позже и большевиков упрекали в том, что ради иллюзорного «общего блага» готовы пожертвовать каким-нибудь конкретным человеком. Многие наши мигранты первой волны объявили Петра первым нашим большевиком. Пётр первый любил себя сравнивать с учитилем, наставником народа. Народ, как дети, ученики мастера, которые не сядут учиться до тех пор, пока не будут принуждены.
Знаменитая идея, патронолистская: государство лучше знает нашу пользу.

Трудно назвать правителя, который не совершал у нас довольно серьёзных ошибок. Сейчас совсем другая культура, образование. Всё равно считается, что государство знает лучше наши пользы.

Ещё идеи, которые были близки к Петру, и она вам тоже понравится. Идея регулярного государства. Идея, подсказанная математиком Лейбницем. Лейбниц сравнил гос управление с часами. (Говорим «как часы» "--- идеал). Идея логичного гос управления. Идея регулярности очень часто у Петра встречается. Петербург "--- первый регулярный город, то есть город по плану. Невский проспект считался символом благоустройства. Пётр сначала хотел центр построить на Васильевском острове. Но оказалось, что это самое низкое место в Петербурге, поэтому центр перенесли на противоположный берег. А Васильевский остров был просто разбит на квадраты. Улицы "--- линии с номерами. Вот пример рациональности, регулярности.

Ещё Пётр старался чётко регламентировать права и обязанности должностных лиц. Между прочим, требовалось, чтобы в присутственном месте на столе должен был лежать список дел нерешённых. Должен был быть докумен о правах от Петра. Была знаменитая фраза: бесталку законы писать, коли их не исполнять.

Обратимся к системе гос управления. При Петре возникает новая система гос управления. Она остаётся ещё сто лет. Пётр получает титул императора. По итогам северной войны сенат попросил Петра принять титул имератора, отца отечества. Это намёк на отцовскую власть в древнем Риме: заботится о семье, но имеет право приговорить без суда своих детей к смертной казни. Термин имератор не вытеснил термин царь.

Второе. Мы говорили, что исчезает боярская дума. Но это было необходимое учреждение. И не очень аристократическое под свой конец. Фактор ошибки в гос делах, цена ошибки особенно велика. И естественно при принятии решения нужно посовещаться с опытными людьми, государственными деятелями. Дума не ограничивала царя. Но если большинство против, начинаешь задумываться. Не может быть большинство избранных (не народа) совсем уж ошибаться.

Вместо исчезнувшей боярской думы (она сама рассосалась, Пётр новых бояр не назначал, старые умирали) появился сенат. Он потом сложным образом преобразовался в верховный суд. Первоначально сенат "--- высший гос совет. Компактный. Человек 10, не больше. Люди, назначенные Петром, которым он доверял. Он не считался с их титулом и родословной.

Известный вопрос: это царское дело или нецарское? Что важным считать? Обо всём писать императору? Эта проблема так и не решина.
Сенат должен был разгрузить императора, отсеив вопросы ясные. Пётр не святой, человек, который ограничен возможностями, например, ему нужно спать хотя бы несколько часов. Было решено, что если сенат решает дело единогласно, то оно решено окончательно и не поступает к императору. Если хотя бы один против, то дело идёт к императору.

Эта коллегия из десяти человек не может всем управлять. Нужны другие органы управления. Появляются коллегии. Взята идея у Шведов. Кругом насаждались коллегии, потому что в единой персоне не без страсти бывает. Чтобы не допустить ошибок (не могу сам за всем смотреть), да пусть будет коллгия. По регламентам решения в коллегии (около 10 человек: ассесоры, советники, президент, вице-президент) принимались так. Если решили рассматривается, то нужно голосова. Сначала голосуют младшие чины. Последним голосует президент. Это было сделано так, чтобы мнение президента не давлело. Если голоса поровну, то у президента голос решающий. А так решения принимались большинством даже против президента.

При Петре возникает современная система гос контроля. Это требовала и практика и теория. Понятие прокурора на не нужно объяснять. А ещё появился фискал. Фиск по латыни казна императора. Главная задача фискала "--- следить за казёным интересом. Если они замечали нарушения, тем более воровство. Они должны были писать тайные доносы и на суди обличать. Сами не должны были вмешиваться. 

Фискалами ещё называли просто доносчиков. Процесс называли «фискалить». При советской власти этот термин заменился на терин «стукач».

Чтобы поощрить фискалов (у них не очень большая зарплата), получаешь процент. И сейчас такое часто. А то и вообще без зарплаты. Деньги суть артерия войны. Но если по твоему доносы делается начёт (штраф), то часть из этого штрафа в качестве премии начисляется тебе. За ложные доносы не наказывали.
Не такая простая схема. Какой смысл брать взятки, если ты можешь честно заработать больше. 

Но кругом была коррупция, естественно. Причин много. Выдвигаются новые талантливые люди. Они желают быстро разбогатеть. Как при Петре воровали до не воровали, разве только позже.

Теперь про функции прокурора. Они должны были готовить фискалов. Фискалы были подчинены прокурорам. Прокуроры осуществляли гласный надзор: чтобы всё было по законам и по регламинту и чтобы соблюдался гос интерес. Могли вмешиваться непосредственно в ход дела. 

Ещё был табель о рангах. Он нам интересен по ряду причин. Смысл простой. Все должности (вначале речь шла о должности, потом уже получилась система чинов\footnote{Командир полка "--- это должность, полковник "--- чин. Можно быть полковником, но не командовать полком.}). Низние чины "--- последний ранг. Высший чин "--- последний ранг.
Причём эти чины были разделены не только по вертикали. Ещё дополнительно разделены на три группы: военные чины (отдельно морские, сухопутные, гвардейские, артиллерийские), статские (то есть гражданские), чины придворные. Возникала такая сетка чинов. И получалось, что какой-то чин, скажем, военной службы имел соответствие с чином статским и придворным.

Оценим этот документ. Это памятник петровской идеи регулярного государства. Во-вторых, когда окончательно менял, это заменяло окончательно принцип учёта знатности принципом личной выстуги. 14-й класс "--- это уже офицер в армии. А сначала надо было солдатом послужить.
Для каждой из ступеней обозначался срок нахождения. На каждой не меньше трёх лет. За знатные заслуги можно просить досрочное производство в чин.
Тогда часто и даже в армии на западе (не у нас) должность и чин можно было продать. Это считалось собственностью. Если есть деньги, сразу покупаешь полковника. Такого в России никогда не было. А что же было?
Постепенное продвижение по служебной лестнице.

Действовал принцип старшинства. Если возникала вакансия высокого уровня. Эту вакансию предлагали тому, кто на более низком уровне находился дольше всех. И так в порядке живой очереди.

Табель о рангах регуляровал порядок получение дворянства. Это показалось через время слишком просто. Тогда возникает разделение понятий дворянин и помещик. 
По чинам обращались: ваше благородие, \ldotst{} ваше высокопревосходительство. И к жене нужно было так обращаться и к незамужним дочерям.
Запрещалось не умеющим чины людям иметь платья с золотым и серебряным шитьём. Всё это делалось для того, чтобы поощрить дворянство на службу (военную).
\subsection{Значение}
Как можем оценить реформы Петра первого. Традиционное слово: неоднозначно.

Я уже говорил, что первые эмигранты назвали Петра первым большевиком. Не щадил себя и тем более других. Как и Сталин, Пётр любил рыть всякие каналы. Он был революционером.
Петра первого нельзя отнести к эффективным руководителям. Ключевский сказал: сколько Россия средств потратила на северную войну, ещё 6 захваченных Швеций не окупили бы. Мы и сейчас не 
эффективны. А это всегда нужно власти предъявлять. Сталин тоже не являлся эффективным менеджером. В отличие от большевиков, у Петра было одно преимущество. Пётр хотел сделать из России Голландию. 
Большевики строили общество никому не известное, книжное выдуманное. У Петра были более реальные цели. Хотел унаследовать достижения передовой цивилизации. 

Зачем с бородой бороться? Излишне людей раздражать? Умелый политик наоборот смягчает реформы. Как врач смягчает сложную процедуру ребёнку конфеткой.
На западе причём в какой-то момент борода была в моде. Разве борода с нашей точки зрения более нелепа чем парик петровской эпохи? Если бы я был сейчас в таком парике, вы бы вызвали скорую помощь.

Во время царской России, все закрепощаются во имя государства. Нет бездельников. Это закрепощение достигает максимума во время Петра первого. А потом начинается раскрепощение. Свободного человека не хватало. 
