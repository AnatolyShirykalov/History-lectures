Политические особенности традиционного общества. Для такого общества характерна либо феодальная монархия. Её виды следующие: 
\begin{itemize}
\item ранне феодальная (как Древняя Русь),
\item сословно-представительская (нас не проявилось). Обычно три сословия
\begin{enumerate} 
\item Церковь пользуется уважением, авторитетов. Имеет землю
\item Дворянство. Защищают свои интересы с оружием в руках. Королевская армия невелика, государь созывает вассалов со своими армиями.
\item Горожане простые могли встать на стены и отстаивать город. Много было вооружённых людец. Город "--- это 
\end{enumerate}
\item Абсолютизм. Переход от традиционного общества к буржуазному. 17-й"---18 века время абсолютизма Франции.
\end{itemize}
При Феодализме были и республики. Была Новогодская, Псковская республики. В Германской империи были вольные города.
Милан восстал против Барбаросы.

Республика была не демократическая. В городах средневековых власть была у элиты: купцы, промышленники. А основное население имело очень небольшое влияние на власть. Нормальная ситуация: 40\,000 населения, 800 имеют право голоса в городской совет.

За всю республиканскую Новгородскую историю (лет 300) управление было только у бояр. Аристократическая республика, а не демократическая.

Традиционное общество, понятно, со временем становится менее традиционным.

Средневековая культура в основном религиозная. Что сохранилось: храмы. ИЗО тоже религиозное, иконы. Образование долгое время было с богословием. Та же Сорбона была основана монахом. Быть человеком разумным в средние века, нужно хорошо знать религиозные тексты. Но ещё в те времена всё больше изучание античных мыслителей. Главным образом Аристотеля.

В восемнадцатом веке в закрытых элитарных школах изучали латынь, древнегреческий. Пушкин вспоминал о том, что он читал в лицее, а что не читал. У него много эпиграфов на латыни. Все лекции полагалось читать на латыни. Екатерина разрешила русское право преподавать на русском в виде исключения. Латынь позволяла студентам перемещаться (но у нас своя судьба). Договоры и трактаты были на латыни.

Итак, общество религиозное "--- это первое.

Общество не развивается. Нет идеи развития, она появилась потом, в новой эпохе, эпохе просвещение. Был взгляд, что ничего не развивается, всё движется по кругу, как сезон работ. Ощущение, что как было при отцах, так и будет.

Другая идея "--- идея упадка. Был золотой век, когда правили и жили боги. Потом был серебрянный век "--- полубоги. И потом уже железный век, бог не вмешивается в жизнь людей. И это идея христиаская и исламская. Помните изгнание из рая. Раньше были старые добрые времена.

Ещё свойство культуры и быта традиционного, что данное общество было авторитарным. Монах более-менее авторитарный. В его руках жизнь, смерть, имущество подданых. Обычно монархи добрые, но все понимали, что он может судьбу решить.

Итак, есть традиции, есть религия. Есть пример Галилея, который отрёкся, а это 17 век. Джордано Бруно зашёл на костёр, упрямый человек.

В семье авторитет родителя, особенно отца. В Римском праве запрещалось убивать детей до пяти лет. Дальше было можно. Такой авторитет отца. Римские нравы, стандарт поведения, без суда, пользуясь властью отца.

В христианские времена за убийство детей наказывали покаянием. Наоборот, особо жестоко каралось убийство матери, убийство отца. Чтобы в брак вступить, нужно одобрение родителей. А в 16 веке родители вообще не спрашивали детей. Можно с другом друг другу пообещать, если у одного будет сын, у другого дочь, что будет брак.

Человек несчастен, когда его представления не соответствуют тому, что он видит вокруг. А если он знает, что власть у монарха, а как же иначе, что надо поститься и так далее. Он будет счастлив. Не испытывали люди дискомфорт.

\section{Основные эпаты истории древней Руси}
\subsection{Девятый, десятый века}
От призвания Варягов, до вступления князя Владимира. 

Свою историю государство наше ведёт от 862 года. Призвания варягов очевидно легенда. Ведь тогда ещё не было письменности. Она появилась лет через сто. Нужно было передавать устно. Неизбежно что-то исказалось. И было другое отношение к истине. Так что это легенда. Но политически и психологически это могло быть.

Новгородцы платили дань варягам. Они их прогнали и сами стали управляться. Случилось восстание и борьба за власть. Что естественно. И тогда призвали новых варягов, чтобы управлять, а не традициями.

В средневековье самый нормальный способ передачи власть по наследство. Как иначе: выборы всенародные? Мы и сейчас их не можем провести. А тогда был выбор либо вечная борьба, либо власть династии, которая оторвана даже от знати. Она не связана родственными отношениями с подданными. Браки только международные.

Брак с поддаными провоцирует зависть других.

И вот пригласили монарха. Это один из способов разрешения конфликта внутригруппового. Пригласить нейтрального судью, чтобы никто не чувствовал себя проигравшим или победившим.

В начале 17 века была смута. Семибоярщина. Приглашают польского Владислава. Один из переговорщиков говорил, что нам не в первый раз пришлашать. Наверно, что-то имело место быть.

Сейчас в Швеции правит иностранная династия.

Греки, когда получили свободу при русско-турецкой войне, вырали президентом русского. Они его убили в итоге и сделали монархию.

И последнее по поводу призвания варягов. Призвали, чтобы правили по праву, то есть по закону. Можно было в средневековье получить титул. Но власть зависела от того, как ты её получил. Самая сильная (и непрочная) монархия, когда ты захватил силой. Ты правишь по праву сильного. Непрочная, потому что не хватает легитимности, появляются конкуренты, мстители.

Или ты получил власть по наследству. Это средний уровень. Ты легитимен, но ты должен увадать обычаи своих предков.

Третий вариант самый слабый, когда тебя приглашают на престол. Преглашают на определённых условиях. Полную власть не отдают. Не согласен, другого поищем. Условия ограничительные. И управление вместе с элитой местной. И варяге так позвали.

Смысл данного этапа, век десятый, "--- это становление государства. Это процесс примерно на сто лет. За это время произошло.

Первое. Формируется территория государства. Захват Олегом Киева. Киев становится столицей, но Новгород давал Киеву князей. Современным языком Новгород "--- северная столица. Опираясь уже на Киев, подчиняют соседние племена. Последнии присоединились племена на месте современной Москвы. Итак, земли восточно-славянских племён (хотя и не только, например, финно-угорские)

Второй атрибут "--- суверенитет. Бывает и промежуточное состояние, как сейчас страны евросоюза: ограниченный суверенитет.
Неприятная вам новость: учитывая сложные отношения с Россией, есть идея заключать с Россией стандартный контракт на покупку нефти газа. Это монополия потребителя. Такая идея подкинута. В евробюрократии много интузиастов. Процесс идёт.
Это к тому, что нельзя говорить, есть суверенитет или её нет. Есть много переходных форм.
Если страна подписывает международное право, это уже ограничивает суверенитет.

По летописи к моменту призвания варягов и захватом Олегом Киева русские платили дань варягам. А Киевляне платили хазарам. И вот теперь они повинности несут в пользу киевского князя.

Третий атрибут: возникает публичная власть (а не в племенах народное собрание, избирают вождём; звание вождя нельзя унаследовать), государство. Власть становится отделённой от народа, власть профессиональная. Отчуждение власти от народа. Оно и сейчас есть. А обширной стране трудно использовать методы прямой демократии, нужно передавать власть народа. Власть состоялась из трёх элементов: князь, дружина и вече.

Не только формируется власть, но и первые реформы самоуправления. Появляется опыт. Я напомню о знаменитой гибели князя Игоря. В 945-м году убит древлянами, муз известной княгини Ольги. Собрал дань, на обратном пути дружинники предлагают ещё пособирать, а то местные мол лучше одеты, обеспечены.
По византийским источникам Игоря разорвали на части. После этого Ольга провела реформу. Здесь информация о древне государстве. Есть ещё местные национальные князья, есть свои дружини. Для местных Игорь "--- это волк, который ворвался в стадо. Даже самый неразвитый человек не любит, когда власть к нему в карман лезет.
В Англии первым упразднилось право Монарха свободно деньги собирать.
И Ольга провела реформу.

Бог в библии (ветхий завет) жесток. Соддом и Гомора, зуб за зуб. Это потом появился другой бог Иисус. Мормоны полагали, что если в библии есть многожёнство, то можно и здесь многожёнство. Для них образец библия.

Итак первая реформа гос. управления. Ольга установила чёткие размеры дани. И во-вторых, установился погосты (места сдачи). Погосты "--- места собрания племени, туда приезжали купцы, гости. А потом слово поменяло смысл на кладбище.

Итак, постепенно Русь приобретает признаки государство.
\subsection{Владимир и Ярослав}
Это время рассвета древней Руси. Это крупнейшая страна в Европе (хоть и не только), это богатое государство. Успешно функционируют торговые пути, которые имели мировое значение.

Считается, что Старая Ладога "--- первая столица. По легенде один из курганов "--- могила князя Олега. Его разрыли и там артефакты более древние оказались.

Можно было по рекам переходить.

Трудно было богатеть на сельском хозяйстве. Богатели на торговле. Она и обогащала древнюю Русь.

Страна единая, единый монарх; а страны европейские уже переживают раздробленность.

Ещё причина подъёма: принятие христианства. Выбор цивилизационный. Европейская цивилизация христианская, а восточная "--- ислам. Владимир предложил представителям презентации сделать.

И христиане и ислам одинаково ненавидели язычников. Долгом христианина или мусульманина была борьба с язычниками.

После колебания, было выбрано восточное христианство. Греция "--- самая процветающая и богатая страна европы. По этой позиции выбор был разумен. А дальше следующее. Византийская (название придумали в 16-м веке, себя они называли Ромеями), на самом деле называлась Восточной Римской империей. После её падения почти одновременно становится незавимое Московское государство, единственное единое крупное государство с православием.

Для православных католики еретики и наоборот. Были приглашения на престол в 17 веке, от которых отказались из-за условия на смену религии.
