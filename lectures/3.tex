Выяснили индо-европейцев, что они делятся на западных индо-европейцев и индо-иранцев. Западные европейцы делятся на романцев, германцев, кельты. Кельтский язык исчез, он не возрождается.

Как раз мы остановились на Кельтах. В прошлом Кельты в чём-то доминировали в Европпе. Древнее население Испание и в Британии они жили. Чехия и Богемия в какой-то степени от Кельтов появилась. Теперь их меньше, Шотландцы, Ирландцы, Валийцы. 

\begin{azItems}
 \item Романцы
 \item Кельты
 \item Германцы. Германия не была единым государством. Она была поделена на маленькие государства, которые различались даже религиозно. Но они смогли объединиться этнически. По политическим причинам некоторые страны не вошли в состав единого государства.
\item Славяне. Письменность же появляется поздно, во многом зависит от потребности в накоплении и передачи информации, появляющейся вместе с появлением государства. Зачем вообще нужно оформлять документы, помнить дату рождения? В точности это людям не нужно. А государству это нужно. И службу записать, кто где служит.

Этногенез славян для нас неясен. Германцы были пограничны с более развитыми племенами Романцев. Не германцы о себе писали, а соседи о них. А вот славяне жили дальше. Римлянам они были малоизвестны.

В течение первого тысячилетия до нашей эры формируются на территории восточной Польшы и примыкающих Беларуссии и Украины самостоятельный этнос Славяне. Небольшая территория. Но славяне участвую в великом переселении народов. И распространяются во все стороны. Становятся самым многочисленным народом Европпы.

Первая версия. Понятие славяне происходит от термина «слово». Словаки и Словени пишут это слово через «о». Славяне "--- группа людей, которые понимали друг друга. Понятно, что сейчас только специалист может определить родство санскрита с русским языком. Гимны назывались ведами, а веды это знания. В русском есть понятия веды, ведомости, ведьмы (обладатели тайных знаний). Агний "--- огонь. Но всё-таки весьма сложно родство раскопать. Ещё в первые века нашей эры славяне говорили примерно на диалектах одного языка. Первые христианские книги писались у нас по-болгарски. Болгары уже были христианами, не надо было переводить, все понимались. Те, которые были менее понятными по языку, назывались немцами, то есть немыми: с ними говоришь, а они не знаю.

Вторая версия. Славане происходит от слова «Слава». Настоящие славянские имена Святослав, Переслав, Станислав\ldotst{} В шестом"---седьмом веках славяне распадаются на группы
\begin{itemize}
\item Западные. В восточной германии жили потомки балавских славян. Немцы тогда ушли на юг и на запад. Германия находится на территории тех славян. 
\item Восточые.
\item Южные. Она позже других возникла. Народы распавшейся Югославии и болгары. Была мысль, неодобренная Сталиным, включить в состав Югославии и Болгарию.
\end{itemize}
Почему сначала раскололись западные и восточные, а потом уже отделились южные. Различия между западными и южными намного больше. Южные "--- это в основном восточные славяне, которые переселились на юг. 

Восточные разделились со временем
\begin{itemize}
\item Русские
\item Украинцы
\item Белорусы
\end{itemize}
Происхождение русских тумманное. Первая версия: призвание Варягов, Скандинавы. Факты известны, ничего нового не нашли. Аргументы за: «мы от рода русского», а дальше нерусские имена. Но против: нет на территории Скандинавии племён с названимем Русы. На время призвания варягов приходится период их максимальной экспансии. Много Скандинавских захоронений под Смоленском. Но это важный узел: верховья Днепра. В этом регионе нашли горшок 9-го века.

Вторая версия: автохтонная, местная. Один из притоков Днепра называется Россь. Древния наши законы назывались «Правда Росская». В этом месте возникает Киевское государства.

Третья версия: компромисная. Смешивается северное понятие Русы, и местое понятие Росы.

Если брать на веру вторую версию, то Украинцы правы в том, что мы узурпировали Русь. 

Ещё один комментарий, который имеет современную актуальность. Когда возникла Украинская национальность? Медленный процесс перехода количества в качество. Многие Украинцы начинают считать, что они не часть русского этноса, а самостоятельный этнос. Я считаю, что это началось во времена Советского Союза.

Союзы черносотенцов были на Украине из Украинцев. Несколько миллионов человек, крестьяне. Они считали себя частью русского народа.

Вот когда дали государство в составе СССР. Белоруссы не просили совсем. Но это провоцирует представление, что вот мы отдельный народ, самобытный совершенно. Последствие большевистской. Даже в Грузии до революции не было сепаратизма. Многие общероссийские партии грузины возглавляли. Князь Багратион хороший известный пример. Меньшевик князь Церителли. Петросовет возглавляет меньшевик Чечевидзе. Большевики не имели поддержки в России, кроме Москвы, Петербурга и армии. Нацианолисты требовали независимости, но не государства. 

Последние события на Украине самобытность национальную украинскую укрепили, даже в восточной части прорусские настроения ослабли.
\end{azItems}
\section{Древняя Русь}
IX--XII века. Киев не был столицей всё это время. Был Новгород, он эже был новый город. А есть старый: Ладога. Рюрик правил в Новгороде. Но потом столица переносится в Киев. Киевская Русь разваливается. Во время нашествия татаро-монголов Киевом правил князь не наш. 

\begin{enumerate}
\item 
Древняя Русь как традиционное общество
\item Основные этапы развития
\item Эволюция государственного строя.
\item Монгольское нашествие и его последствия.
\end{enumerate}
\subsection{Как традиционное общество}
Относится к типу традиционных обществ. При Марксизме была формационная теория, но применима только к Европе. Восточные государства развивались своеобразно. Делали рывок до нашей эры, а затем мало что менялось. Обычно история таких государств идёт по династиям, потому что развития нет особого. По большому счёту управление государством в Китае в конце 19-го века происходило так же, как две тысячи лет назад.

Сам Маркс ввёл понятие восточный способ производства. В моду вошёл подход цивилизационный. Больше похож на восточный. Но я вам предлагаю простое деле не два этапа: общество традиционное и общество современное. В чём разлиция? Древняя Русь относится к типу традиционного.
\begin{enumerate}
\item Различия в экономике. Традиционное общество сельскохозяйственное преимущественно. Почти монопольный вид хозяйства. Современное общество (буржуазное, ещё 18-й век; хотя некоторые общества и сейчас традиционные) "--- общество промышленное. Вот очевидное различие. Сейчас есть новые непонятные пока тенденции. Сфера услуг главная. В Америке идёт реиндустриализация. Раньше модно было производство в Китай переносить, а сейчас возвращают. Рабочая сила дорожает, появляются претензии к правам. Есть понимание стратегий конкуренции. Для России это "--- задача, но она, как и другие, нерешаемая. Есть сырьевая экономика и сфера услуг, а промышленности нет.
\item По месту жительства. Люди в традиционном обществе жили в сельской местности. Города "--- это ограждённый пункт. Ставка князя, центр торговля. Городов мало и они ничтожны. Сельская местность же "--- это большая деревня. Помню схему, что в средневековье три четверти европейской территории покрывали леса, в среднем средневековье "--- половина. Теперь уже четверть. Современное общество городское. Процесс урбанизации универсален. Возникает обратная общества. Тогда почти все в деревнях, теперь почти все в городах. Даже то, что можно жить в отдельном доме загородом, "--- по уровню жизни уже город и это не крестьянское население. 
\item Социальное по структуре общества. Это различие не сразу возникло (не в первобытном обществе). Раньше сословия, теперь классы. В чём разница? Люди в сословия попадали по происхождению. Чтобы упростить, сословия делились на две категории: знатные и простолюдины. Между прочим, знать была в основном военной. Почему? Военное дело "--- самая важная профессия (важная часть формирования феодализма в Европпе). Для безопастности признавали власть над собой. Безопастность "--- самая важная потребность человека после питания. Военное занятие поняли. А как оплатить это занятие? Дефицит звонкой монеты. Монета стоила столько, сколько стоил металл, из которого она изготовлена. Не было принудительной стоимости. Монету можно было разрубать на части. Деньги больше в форме овала. Когда не было серебра, раздавали землю. Так помещики у нас возникли. Постепенно военное сословие отрывается от остальных людей. Становится неприято вступать в брак с человеком другого сословия. Родословную неизменишь, даже деньгами. Потом уже появилась практика продажи титула, это уже разложение, деградация феодального общества.

В России до Петра Первого был только один титул Князь. Они не продавались и не жаловались (не раздавались). А после буржуазия появилась из крестьян за военные заслуги.

Сословия делятся на превилегированные и непревилегированные. Есть элитарные университеты, закрытые школы. Не всех туда возьмут. Возникают связи знакомства по всему миру. Тебя везде устроят. Сейчас это всё есть. Но формально перед законом все равны. А во времена сословий неравенство было закреплено законом. Только дворяне имеют право владеть крепостными, купцам уже нельзя.

Классы делятся по занятию, по професии, по отношению к собственности и в конечном смысле по богатству. Классы "--- это бедные, средний класс и богатые, если упростить. Более развитые страны те, где господствуют великий средний класс. Это обеспечивает общую стабильность, о чём писал ещё Аристотель. Это одна из причин, почему партии на западе очень похожи. Они все ориентированы на средний класс, это основные избиратели. А проблема России в том, что разрыв большой, слабый средний класс. Есть понятие «новый бедный». Высшее образование, профессия, а получает мало. Учитель в провинции, врачь. А есть ещё пенсионеры. Во многих странах пенсионеры очень счастливо живут, путешествуют. У нас не так. У нас трудно жить.
\end{enumerate}
