\section{s}
Мы обсуждали исследования, их особенности, сложности исторического исследования. Прийти к точным выводам в истории сложно. Процессы сложные, социальные, у людей мотивы бывают трудно объяснимы, эмоциональны, трудно предсказуемо. Невожможно доказать, что твои оценки верны.

Существует известный афоризм: «история не знает сослагательного наклонения». Могла быть альтернатива. Редко бывает, когда движение в заданном направлении неизбежно. Вот в России всё так и должно быть? Или есть альтернативы? А ты докажи, что может быть лучше, вот докажи!

Трудно провести грань между прошлым и настоящим. Начало лекции "--- уже прошлое.

Главный принцип "--- подход диалектический. Плюсы и минусы. Одна из свойств политики или человека в этом случае может быть хороша, а в другом плоха. Известный метод: кнут и пряник. А вот в конкретном случае, что лучше использовать: внут или пряник? Санкции "--- это хорошо, плохо? Ну как ответить?

Сейчас кто-то ссылается на то, что западные страны обожглись на Гитлере в Мюнхене. Поверили, он же обещал. Намекают, что Путин тоже такая фигура непредсказуемая. Я думаю, нет.

Вернёмся к абстракциям.
\begin{enumerate}
\item Диалектика.
\item Принцип полноты исследования. Закон достаточного обоснования. Нельзя заниматься тем, что (Ленин не авторитет, он сам этим занимался, но слово хорошее) игрофактикой. Нужно как можно больше фактов. Но я уже говорил, что тут сложность. Нужно остановиться и не утонуть в фактах. Нельзя искуссно факты группировать. Пропаганда как раз этим занимается. Есть факты, которые я использую, есть факты, которые я игнорирую. Или за или против.

Как говорится, факты упрямая вещь. «Ваша теория противоречит фактам» "--- «тем хуже для фактов».

Ошибки всегда есть. Если вернуться к России, то найдите чистенькую страну? Вопрос в мере, кто более нагрешил. В чём проблема России современной? С одной стороны очень старая страна, ведём историю от призвания Варягов. Но с другой стороны, мы страна молодая. В этом качестве и в этих границах мы живём только 20 лет. И эти проблемы относительно Крыма и Донбасса проявились. Другие страны эти проблемы уже давно пережили. И политические проблемы пережили и экономические.

Один из величайших американский президент Линкольн был во времена рабства. Тянул с отменой рабства года полтора, желая достичь компромисса с южными штатами. Главное было сохранить целостность США. Южные штаты считали, что добровольно вошли, добровольно и выйдем. Самая кровопролитная война в США была за целостность. А если Горбачёв затеял такую за сохранения единства? В общем такого рода проблемы для других стран решены были давно.

Итак, полнота исследования, и нельзя искуссно подбирать факты. Все нужно учесть, не просеивать.

\item Историзм (или конкретный исторический подход). Что это такое? Формулу не запишешь, нужно объяснить. Нужно помнить, что в принципе в смысле антропологии люди за последние несколько десятков тысяч лет мало изменились. Но с точки зрения моральных ценностей и правосознания изменились сильно. Есть конечно заповеди, не убей, не укради. Но они и сейчас не универсальные. А если напали на тебя? Закон гарантирует необходимую оборону.

Веками и даже христианскую эпоху самые славные люди "--- великие войны. Военная профессия "--- самая почётная, и самая, собственно, нужная. Помните сказания о княгине Ольге? Ольга "--- одна из первых христианок, почему она действовала из принципа пробной мести? Общество иначе бы осудило. Есть такая версия, одна и причин упадка (очевидно, не главная) падения Рима "--- появление христианства. Раньше ходили убивать, а теперь убивать это грех. Мораль вообще деморализует.

Человек плохой, интриган часто побеждает порядочного. Ведь порядочный относится к непорядочному порядочно. Играет по правилам, по морали.

Итак, в том то и дело: оценивая события прошлого нужно учитывать, что люди знали, чего не знали. Они не знали следующие события, не знали, к чему приведут их действия. И второе, нужно иметь в виду, что меняются мораль и правосознание.

Англий 20-е, 30-е годы. Убил кого-нибудь, намеренное убийство грозило смертной казнью. Теперь посчитали это жестоким, нет смертных казней в Европе.
В конце 18-го века запретили «казни особой жестокости».
После вьетнамской войны, верховный суд США посчитал, а не является ли смертная казнь неприемлимой.

В штатах есть разделение компетенции штатов, компетенции федеральной. Приговор о смерной казни утверждает губернатор. Посчитали, что смертная казнь является жестоким наказанием. Но лет через 7 вопрос пересмотрели и оставили на рассмотрение штатов. Раз в 10 лет какой-нибудь штат отменяет смертную казнь.

Это изменение правосознания. Раньше было сожжение на костре ненормально, а повешение нормально. А вот теперь и вообще казнь запрещают.

Дакентосская мораль: Если я украл корову "--- это хорошо, если у меня украли корову это плохо. Так и Россия сейчас жалуется, используя такую мораль. Это двойные стандарты. Запад тоже так. Что нам выгодно хороша, что невыгодно "--- плохо.

Вы знаете афоризм: он, конечно, сукин сын, но он наш сукин сын.

Помните, что Косово объявили независимым, а потом бомбили. А здесь ополченцы, сепаратисты сами себя взрывают. Нечуствительность населения к этим событиям удивительна. Сукин сын, но наш сукин сын.
\end{enumerate}

\section{Происхождение славян}
Переходим к проблемам более-менее чисто историческим. Этносы, их основы возникали в древности. Очень часто в период дописьменный. На какой стадии развития появляется письменность? На стадии появления государства. Законы писать, налоги собирать. Почему крестьяне всегда были безграмотные? Навыки передаются от отца к сыну, не по иструкции.
С одной стороны не могли позволить годами своим сыновьям учиться в школе. А с другой стороны века пахали поле и ты проживёшь.
Вот почему госудаство провоцирует образование. Получается, государство в в России появилось где-то в 9-м веке.

Славяне были дальше от античного мира, а Германцы были близко. Постоянные завоевания, походы. Более цивилизованные народы описывали Германцев, но не сами они.

Если нет письменных источников, то как изучать оценивать? Чем меньше данных, тем больше фантазии.

Первая наука, хорошо известная, на этот счёт "--- археология. Монеты, пуговицы старинные можно найти. 

Итак, археология. Для этногенеза она не так полезна. Разрывая захоронения древние, ну вот череп и так далее. Занятия можно найти, чем занимались, там скотоводство, уровень технологии можно оценить, чем женщины занимали, там бусы. Но происхождение трудно определить, потому что антропологические типы очень близки. Есть доминирующие типы, там форма челюсти, но тяжело. Можно выяснить когда жил, что делал, чем делал, степерь безопастности. Но вот на каком языке эти люди говорили, можно только гадать.

Этнопологическая культура. Видно, что в нескольких областях жили племена, которые вот этим занимались. Все покойников сжигали, а вот в другом месте хоронили в яму. А третьи в катакомбы. Но что это были за люди, какое было самоназвание?

Вторая наука называется (ещё менее ценна) топонимика. Это изучение истории чезер географические названия. Названия более устойчивые. Новые народы приходят и могут исказить названия, но меняют редко. Только адаптируют к своей фонетике (сейчас только меняют из вредности). Москва, Волга "--- реки, названия финно-угорские. Это названия не Славянские, очевидно, это названия не говорящие, как, например, Белая река. Днепр "--- иранские корни.

А названия американских штатов? Ничего интересного. Многие штаты имеют имена индейсткие. Массачусетс "--- это название индейские племена, Миссисиппи, Миссури. Французская колония была, назвали Новым Орлеаном. Есть и латинские названия, завоёвывали. Американские люди счастливы, им уже нечего завоёвывать. У них нет проблем границ несправедливых, искусственных. Названия сохранились, напеример, Новая Мексика, река Колорадо.

\subsection{Сравнительная лингвистика}
Сейчас у исследователя этой проблемы появился очень мощный инструмент. Какой? Мощный научный, точный. У тех, кто исследует миграцию, происхождение. Иструмент "--- генетика. Если вы разрываете захоронение, можно определить к какому этносу относится на уровне нестрогом. Смешения генетические и мутации происходили всегда.

Вернёмся к сравнительной лингвистике, которая развивалась в веке 19-м. Что это такое? Сравнение языков. А зачем? Считается, что язык является важнейшим критерием этнической пренадлежности. А вообще в Шотландии мало знают кельтский язык. Есть полуумершие языки. Если языки похожи, то значит народы родственные. Специалисты могут найти родство. Давние общие корни.

Вывели понятие: языковая семья, язовая группа. Языковые семьи:
\begin{enumerate}
\item финно-угорская. Кто ей принадлежит? Финны, Венгры (самоназвание Мадьяр), Кареллы (наших мало осталось), народы севера, те же Коми. Некоторые народы поволжья, эстонцы. Народы, которые были ассимилированные славянами: меры, мурома. Пришли славяне и с ними смешались. Русские "--- это смешение славян и финно-угров. Было постепенное вековое смешение. Этот факт можно констатировать. Названия обсуждали сегодня, яковская культура. По месту обнаружения, не по самоназванию. Коломенская "---  любимая загородная резиденция.
\item Тюркская группа. На базе такой структуры Тюркский каганат. Азербайджанцы, Татары, Башкиры. А в прошлом были тюрками были не хазары, а печенеги и половцы, кочевники. Сейчас самый крупный тюркский народ это турки. И даже в прошло на рубеже 19--20-го века был пантюркизм.

\item Индо-европейская семья. Почему такое название? Потому что этой семье принадлежат большинство европейцев, не все. Если брать коренные народы России до уральских гор, то индо-европейцев большинство. Народы Ирана, включая таджиков, народы Авганистана и Индии тоже находятся к этой языковой группы. Первыми в истории зафиксировались народы северной Индии и Ирана. Это народы, которые проникли на север индии в 1700-х годах до нашей эры. Культура Хараппы и Махинджидары (современные названия мест, где эти народы обнаружены).

База отнесения к этой группе язык санскрипт. Самоназвание у них более известностое "--- арии. У арийцев был популярный символ свастики.

Лет десять назад на Волоколамской зашёл в музей. Там находки 10-го, 11-го веков. Некоторые украшения имели символ свастики. Свастика "--- нейтральный символ. Да, его фашисты опорочили, запрещён почти везде. Но сам по себе нейтральный. Арии "--- понятие, отнесённое к иранцам и к северной Индии. Иранцев называли Персами. Это была доминирующая нация. В какой-то момент потребовали называть Ираном, это было на том же уровне, что и Пётр Первый бороды запредил. Иран переводится, как страны арий. Страна Ария.

Индоевропейцев в 19-м веке называли арийцами. Индия сейчас самая пёстрая страна в мире, наверное.

Для нас важнее европейцы. Считается, что около 4\,000 лет до нашей эры индоевропейцы составляют компактный и не очень большой этнос. Откуда они появились. Есть споры, то ли южный Урал. Или это промежуточное место. Они были кочевниками. Раздилились. Одна часть пошла на восток, другая на запад. И сейчас "--- они большая часть Европы.

Кто из них получился
\begin{enumerate}
\item Романцы (язык тут не соответствует этносу, романские языки произошли от литинского: испанский, французский и даже румынский). Энтически французы, испанцы, португальнцы "--- это Кельты, будем говорить. Сильнейшее влияние от латыни, это заметно.
\item Группа германских народов. Немцы, австрийцы, скандинавцы, германцы. У австрийцев и немцев язык общий. Но повороты истории сделали диалекты немецкого языка. Разные племена имели свои государства со своей историей: Бавария, Пруссия, Саксония. Немцы со временем разделились по религиозному признаку и воевали. Все считали, что германия распадётся. Вектор пошёл так, что ядро Германии сохранились.
А автрийцы? Самый известный австрией "--- Гитлер. После проигранной войны, так получилось, получили независимость.

\item Кельты.
\end{enumerate}
\end{enumerate}
