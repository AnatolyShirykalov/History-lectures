
Мы остановились на последующей раздробленности после расцвета древней Руси. Нормальный средневековый сценарий. Причины расцвета в том, что многие европейские государства переживали раздробленность. В чём причины.

Первое. Экономическая причина, отсутсвие единого рынка. Торговля в древней Руси, конечно, была. Внешняя:  мы говорили о двух путях. Внутренняя: много локальных рынков не связанных между собой. В город приезжают купцы и обслуживают местный рынок. Были нужны металлы, топор купить в соседний город. И соль была нужна, потому что это был очень важный способ консервации, не было холодильников. Одна из причин, почему рынки не контактировали: дороги. Вторая причина: а чем торговать? Разделение труда было разделением климатическим, а у нас одна полоса, везде то же самое. Итак, не было экономического интереса в единстве государства.

Непонимание настоящего есть незнание истории (эпиграф к какому-то фильму). Общий рынок превращается в евросоюз. Экономика диктует. А когда распадался СССР, важной причиной было отсутсвие экономики рынка, был план. Если была бы экономика, распада бы не было: раньше могли сказать «не будем Москалям сало давать, нам самим надо», а теперь их на рынок не пускают "--- обижаются. Это другая психология уже.

Вторая. Политическая причина. Древняя Русь "--- крупнейшая европейская страна, при этом с плохими коммуникациями. Во время распутицы сообщение просто прерывалось. По рекам в основном передвигались. Передвижение зимой проще. На санях. Дороги чисто зимние бывают.

Граница Руси проходила чуть ниже Киева. Политический центр был почти на границе. И в нужных центрах нужно было ставить наместников: сыновей чьих-то. А те хотят независимости.
Итак, невозможность управлять даёт наместников, а наместники выходят из-под контроля. Рюриковичам повезло: род оказался очень плодовитым. Каждый из них претендовал на кусочек земли.

Не было страшной внешней угрозы, которая могла бы сплотить. В Московские времена уже было не так. В период раздробленности проблемы на границе "--- проблемы Киева. До Новгорода уже ведь не дойдут. Понятная психология. Такие же факторы разложения русской армии в 17-м году.

Ещё причина: социальная, точнее, позиция элит. Та местная элита поддерживала князей в их сепаратизме. Почему? Во-первых, кому охота платить в столицу. И какой толк от Киева? Он нам поможет? Киевское войско полгода будет идти. Киев "--- это паразит. И когда князь говорит не будем платить налоги, все его поддерживают. И опять же психология: лучше первым здесь, чем сто первым в Киеве. Маленькое, но своё. Так ведь и бояре рассуждают.

Культурно-психологическая причина. Ещё не возникли нации. Мы не русские, мы новгородцы, мы не русски, мы смоляне. Сейчас было бы странно, если бы соседние области воевали. Первично: вот моя земля, мои интересы, а единство вторично.

Распад не был абсолютным. Распад СССР был расплад абсолютным. А тогда такого не было. Древняя Русь "--- страна многонациональная, славяне, финно-угры\ldotst{ } Но основная масса населения состояла из восточных славян. Единая вера была второй причиной того, что распад не был абсолютным. Славяне были православными. Причём церковь формально сохранила единство. Во главе один на всю Русь Митрополит Киевский. Потом был Владимирский, потом Московский.
И последнее. Династия Рюриковичей разрастается, но возникла такая ситуация. Не возникло местных династей. На землю претендовали только Рюриковичи. Это тоже фактор единства. Эти факторы помогли потом эту раздробленность преодолеть.

\subsection{Главные элементы власти древней Руси}
Их три. Первый это князь. Древняя Русь "--- это ранне-феодальная монархия. Монархия тогда вообще преобладала в мире. Редко были республики. Не было внятной альтернативи: либо монархия, либо олигархия. Почему олигархия была хуже: «лучше иметь одного монарха, чем восемь»; они вражновать могут, мы будем раболепствовать хуже прежнего.

7 миллиардов на развитие демократии в Украине на Американцев. Ага, вот она рука Американцев. А Медведев заявил, что 200 миллиардов потратили. Так вы куда их потратили? И кто эффективно работает?

У монархии два преимущества тогда. Более-менее понятно, как власть переходит. И всё-таки наследников, князей готовят к своей участи, их учат. Наместники "--- это тоже опыт управления.

Второй элемент власти это дружина (в украинском дружина значит жена). Почему же она важный элемент власти. Глвная причина это то, что до середины 19-го века, военная служба повсеместно привлекала лучших людей, считалась более пристижной, была в чём-то самой выгодной. Дворянство "--- это всё военное сословие.

Напомню, как формировалась европейская аристократия. Войны везде нужны. Как их оплачивать? Серебра мало, дают землю. Постепенно они начинают считать, что люди, которые живут на их земле, им принадлежат. А на западе действовал принцип «нет земли без сеньора». На Руси были. Куда бы ти ни уходил в Европе, везде был сеньор. Знатный человек в истории не исчезал: замки, анналы истории, служил государю, отметился. Поэтому появляется понятие чести.

 Первая функция князя военная. Все русские монархи до Николая второго на портретах в военной форме. И на публике появляются в военной форме. Дружинники лучше воюют, когда за ними князь наблюдает.

Третий элемент власти это вече. От понятия вещать. Место, где говорят, почти парламент (от термина французского «говорили»). Итак, вече возникли с незапамятных времён. Были племенные вече, орган прямой демократии. Затем вече переходят в города, но сельские вече не исчезли, они до двадцатых годов продержались. Городское единство, мы знаем, довольно искусственное.

Вече собирается в двух случаях. Первый: власть, князь собирается что-то сообщить. Когда киявлян крестили, их собрали, объяснили, окрестили. Нужна князю поддержка. Поддержите меня, отплачу льготами. Второй: вече самосозывалось. Во многих местах был вечевой колокол, набат. В этом случае вече может превратиться в центр народных волнений.

Тогда все привыкли более-менее убивать, у всех было оружие. Хотя бы животных. Это объясняет, почему у европейских феодалов охота "--- самое лучше развлечение. Максимально близкое к военному делу.

У всех было оружие, был топор, рогатина. Не было у всех полного комплекта снаряжения. Но впринципе люди были вооружены. То есть не так как теперь: один безоружен, а другой на танке.

Когда было единое государство, доминировал князь, авторитет, дружина, казна, престиж. А когда страна распадается, падает значение князей. Это девальвация. Один князь на страну это одно, а когда их сто "--- другое. Напротив, значение дружины и вече возрастает.

Между князьями начинаются усобицы. Тогда люди были воинственными. Тогда напасть, захватить и ограбить "--- это доблесть. Самые крупные политики есть завоеватели. Это первая причина. А вторая: непонятно, кто имеет право на престол. Была непонятная лесничная система. Но она запутанная и всегда лихой энергичный молодой князь не желает стоять в очереди возьмёт и захватит. А усобицы ставят князей от дружины и народа. Вторая причина почему вече становится значимее. Князь располагал профессиональной дружиной, а ещё вторая сила это вооружённое ополчение народа (называлась тысячей). Первый боярин именовался тысяцким. Чтобы поднять ополчение, нужно было объяснить, зачем нужна война. И горожане не обязаны князя поддерживать и навлекать на город бедствие войны.

\section{Московское государство}
Этапы развития и система государственного управления и особенности политической культуры.

Я не хочу всё списывать на монгольское нашествие, но понимаете, общество иногда переживает страшный стресс, последствия сказываются на дальнейшем. Возникают традиции, культура.

Исмотрите, опричнина длилась лет семь. И она тоже была стрессом. И Монголы и Пётр Первый и Сталин оставили отпечаток: гипертрофированное государство и неуважение к простому человеку.

Какие же этапы можно выделить в Московской истории.
\subsection{Возникновение и возвышение Москвы}
1147 год "--- условная дата основания города. По поводу Москвы запись приглашения на пир. Москва уже есть. Но нужна дата. Более точную дату не имеем. Дата показывает, когда Москва стала городом, достойным встречи двух князей. Как поселение Москва на территории Кремля существовала уже веке в десятом.

Первый князь появляется в Москве Даниил, младший сын Александра Невского, уже после смерти отца. Каждый из сыновей должен был получить свою часть земли. Старший лучшую, младший худшую. Как и Юрий Долгоруки, Даниил получив плохую землю от этого только выйграл. Почему:
выгодное положение, везение и ловкость первых московских князей (бестолковый и хорошее наследство транжирит)
