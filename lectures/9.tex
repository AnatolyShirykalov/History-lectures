\section{Дворцовые перевороты}
В 1725 после смерти Петра Первого. Это начало. 1762 год конец по одним взглядам. Другая версия заканчиваются 1801 годом. Период этот ассоциируется с частой сменой властью, нестабильностью и ещё инерциальность власти. Не было испульса развития, заложенного Петром Первом. Это было из-за того, что правили женщины и дети. 

Начался определённый период стабилизации. Вторая модернизация "--- это политика Екатерины второй.

Почему после смерти Петра именно?
\begin{enumerate}
\item Из-за самого Петра. Он издал указ о престолонаследии (1722). Он гласит о завещательном порядке престолонаследия. Почему такая странная ситуация? Она будет связана с неопределённостью, которая будет в семье Петра Первого. Сын от первого брака Алексей был символом антиреформаторского развития, антизападником. Старое боярство делало ставку именно на него. Отношения у отца с сыном не сложились. Нарыжкина (мать Петра Первого) сама сказала, на ком Петру жениться. Поэтому и к сыну и к жене Пётр относился холодно. Алексей не был особо лидером. Наследник по мужской линией окажется мёрвым (не без согласия Петра).

Всего от второй жены Екатерины было 12 детей.  Выжили две дочери. Последний ребёнок был сын. Через год после рождения он умирает. Пётр мог передать престол только женщинам. Психологическая травма детства противоборчества со своей сестрой Софьей, довлела над Петром. Пётр уже в 22-м году не доверял и своей жене. Есть косвенные сведения, что он её заподозрил в адьюлтере. Он почему-то не оставляет завещание в пользу Екатерины. Говорят, он даже подумывал о том, чтобы передать престол ещё не рождённому сыну своей дочери Анны. Так или иначе в 22-м году он понимает, что нужно внести ясность. Говорит, кому решу, тому и назначу.

Историки медицины говорят, что Пётр не был очень здоровым. У него были проблемы психологические и даже психиатрические. Говорят, что у него был высокий рост и малый размер ноги (37-й), проблемы с печенью. Умирает он от простуды. В последние часы попросил бумагу и перу, написал «отдать всё», но не написал, кому.

Последней воли нет. Таким образом, своим указом Пётр заложил бомбу замедленного действия.
\item Опять же вторая причина связана с Петром Первым. Реформы Петра раскололи общество на две, как говорил Ключевский, неравные части: дворянство и боярство. И дворянское сословие было разделено по принципу поступления на службу и по принципам восприятия реформ. Отмена местничества (в 81-м году Фёдором) не была ещё забыта. Одна часть дворянства будет следовать традиционным правилам. При Петре возникает новое дворянство. Оно возникает в начале северной войны, когда возникают гвардейские полки. Табель о рангах появляется.

Итак, дворянство раскалывается на старое и новое дворянства. Последние очень недавно продвинулись. И те и другие были после смерти Петра озабочены. Одни, чтобы вернуть, другие "--- чтобы сохранить.
\item Роль гвардии. Гвардейские полки будут занимать особое место в истории 18-го века. Они были сформированы из крестъянских детей. Они показали свою боеспособность по сравнению с наёмными полками. Для Петра гвардеец был идеальным государственным служащим. Для Петра государство было машиной, которая должна бесперебойно работать. Бюрократия возникает именно при Петре. При Петре от бюрократии не требовалось инициативы, они должны были быть прежде всего хорошими исполнителями. Пётр мог поручить даже строительство заводы, и он мог быть уверенным, что это будет выполнено.

Савёлов провёл количественный анализ переворота (вступление Елизаветы Петровны) 1741-го года: 30 \% из крестьян, 30\% из дворян, \ldotst{ } почти все сословия были задействованы. Участники становились гвардейцами и теряли связь с прослойкой общества, из которого они вышли. Говорит Савёлов, что Елизавета была выбрана именно народом. Волеизъявление людей, которое освободилось из земских соборов.
\end{enumerate}

Итак, причины, как минимум три. Их достаточно, чтобы начались дворцовые перевороты. Меншиков предлагает Екатерину Первую. Внука Петра Первого (сын убитого Алексея) старые дворяне. Решила всё гвардия. Элемент устрашения подействовал (если не Екатерина, то мы всем головы порубим) и императрицей провозглашается Екатерина в 1925-м (41 года Екатерине).

По одной версии Екатерина шведского происхождения, по другой латыжка. В 1702 выходит замуж за латыжского трубача. Потом она у меньшикова знакомится с Петром, становится его боевой подругой. В 1705-м году родила первого сына, затем второго, оба умирает. Потом Анну и Елизавету.

Для Петра Екатерина была идеально такой партией. Брак так партнёрский. Екатерина его везде сопровождала. Даже в походы. Играла немаловажную роль в заключении мира с Турцией, отдав свои драгоценности. Все тяготы походной жизни разделяла.

Во-вторых, что касается образования "--- разные мнения. Видимо, азы грамотности знала. Говорила, кстати, на шведском. Глубокого образование естественно не получила. Обладала житейской мудростью. Умела благотворно влиять на Петра, когда у того были вспышки ярости. Только она его успокаивала, он засыпал, просыпался, как ни в чём не бывало. Сохранились письма немногочисленные, но неясно, она ли писала. По этим письмам отношения у них были до конца достаточно доверительные.

Итак, Екатерина становится императрицей. Главным советником становится Меньшиков. По его совету создаётся Верхный Тайный Совет. Понимают, что будет стабильнее туда свормировать правительство из старого и нового дворянства. Можно сказать, что Меньшиков в течение двух лет и управлял собственно государством. 

Кроме того, Екатерина проводит популисткие меры: отмена смертной казни, снижение налогов. Это будет скорее в теории, чем на практике.

Екатерина в 1727-м умирает. Оставляет завещаение. На престол вступает Пётр Второй. Ему было 11 лет. (Родился в 15-м.) Он не воспитывался с матерью. Воспитывался в семье родственников Петра. Мать Петра умирает. Воспитание он толком систематическое не получает, образования нет. Вступает на престол.

Есть мнение, что именно Меньшиков советовал Екатерине оставить престол Петру. Он хотел закрепиься у престола. Пётр переезжает к Меньшикову, тот называет его отцом. Но воспитателем назначили немца Остенмана, который имел зуб на Меньшикова. Меньшиков заболел, в этот время Остенман его обработал. И более Пётр Меньшикова не признавал.

Пётр Первый ещё сдерживал Меньшикова, но потом Меньшиков сильно злоупотреблял казнокрадством. Его в этом обвинить было нетрудно. Его Пётр Второй Меньшикова отправляет в ссылку.

Объявляют помолвку с дочерью князя Долгорукова. Происходит смещение влияния в Верхновном Тайном Совете в сторону старого дворянства. 

Свирепствовала эпидемя оспы, Пётр умирает. Долгорукие попытались сфабриковать завещание своей невесте. Но не прошло. Верховный Тайный Совет понимает, что из законных наследников есть Елизавета Петровна (дочь Петра Первого), но она была рождена до брака. В ней видели опастность возвращение старых Петровских порядков (по её характеру). Пётр Второй был к ней, кстати, не равнодушен. Дворянство делал всё, чтобы такие отношения не сложились.

Поиск кандидатуры приводит в Кулряндию (Латвия ныне), взор падает на племянницу Петра Первого (было же двоецарствие с Петром Первым) Анну Иоановну. В 1696-м году Иван умер и Пётр правил единолично. Дочери воспитывались матерью единолично. Эта мать была мудрой женщиной, она сохраняла хорошие отношения с Петром. Пётр был прагматик, и выдал Анну за Кулряндского Герцога. Но Герцог сразу умирает, Пётр ей сказал сидеть Анне в Курляндии. Бедная женщина остаётся в Курляндии. Это был медвежий угол. Денег нет, полностью зависит от Петра. В одном письме просит платье, мол даже в свет выйти не могла. Старалась ни с кем не общаться, чтобы не вышла замуж. Он откровенно скучала.

Если до 30 была приятная внешность. Потом богатырская фигура. В часы досуга палила из окон по воронам и курам. Когда же ей предложили стать императрицей, скорее всего она даже не дочитала. Потому что в письме были условия, кондиции. А условия были очень жёсткие. Она не имела права заключать мир, объявлять войну, выходить замуж. Ей уже было всё равно, бедной женщине.

Пётр второй успел перенести столицу Москву. И вот была процедура возведения в императорский чин. В процессе подготовки группа дворянства вышла на Анну и сказали, что не обязательно подписывать кондиции, что её поддержат многие. Кондиции "--- попытка изменить государственный строй, ограничить монархию. Анна во время коронации подписывает и разрыват. Верхный Тайный совет разрываюся. Говорят, что молодцы дворяне, которые спасли Россию от прихода Верховного Тайного совета к власти.

Но всё не так однозначно. Анна была не сильно дееспособна. И нужен был орган управления. И второе, всё-таки был шанс отойти от абсолютизма.

Итак, верховный тайный совет был распущен. Анна правит 1730--1740. Её правление часто называют биронщиной. Возглавляют важные структуры люди не с совсем славянскими фамилиями. В Академии наук были только одни немцы, один русский Ломоносов.

Раз совет распускается, нужен новый огран. Кабинет министров. Те же полномочия. Сама Анна Иоановна была неглупая женщина, но возникает подозрение, что она засиделась в Курляндии и решила брать от жизни всё. Шуты, карлики, сказочницы, балы, маскарады. Иностранцев поражал блеск русского императорского двора. При Петре дворянство раскололось на двое. Теперь появилось третье, и те старые и новые объединяются, консолидируется с целью противостояния внешнему врагу: засилием немцев. Ограничичвается срок службы до 25 лет. Раньше срок службы не имел места.

Были подозрительности, слежки. Легко было обвинить человека в непринятии Анны Иоановны. Арестовывали очень легко.

Анна своим наследником делает сына своей племянницы Ивана Антоновича. Регентом назначается Бирон. Надо сказать, что при Анне активизировалась внешняя политика. Улучшились русско-английская политика. 33--35 год в Европе начинается борьба за Польский трон. Россия добивается прихода к власти Августа Второго. Начинается русско-турецкая война. Особых сражений первое время на было. В 29 подписывается Белградский договор. Россия получает Азов, но не имеет право иметь там флот.

Итак следующий переворот 1741 год. И до 61-го года будет у власти Елизавета (внебрачная дочь Петра Первого).
В заговоре Разумоский, Александр и Пётр Шуваловы. Гвардия поддерживала. Пришла в казарму: «Вы знаете, чья я дочь, идите за мной».
Иоана Антоновича заточают. Потом будут попытки освобождения. Мать его умирает.

Елизавета из народы научилась любить русские обычая. От иностранцев знала языки, но хуже было с наукой. Не знала, что Великобритания "--- это остров. Прекрасные внешние данные у неё были. 
Анна Иоановна следила, чтобы Елизавета не вышла замуж. Ссылала женихов. Чтобы не родился вдруг сын.

Елизавета хороша была собой. Были фавориты. Шувалов, например. Он отказался от всех чинов, жалования. Оставил важный след в русской культуре. Был первым куратором Московского Университета. Именно через него Ломоносов добивается открытия университета в Москве.

Было не принято надевать одно и то же платье. Тысячи башмаков.

Немецкое прибалтийское дворянство было неспеша отстранено. Кабинет министров был сразу распущен. Создала свой орган. Самое главное, что она делает: восстанавливает в правах сенат. Сенат правда находится под контролем императрицы. Первый месяц правления: 41 указ дала сенату, 14 докладов на подписание от сената получила. Она пересматривала очень много дел от сената. Так что нельзя сказать, что управлением не занималась.

Нужно было писать законодательство. Сенат не справился с этим заданием. Важным достижением будет отмена внутренних таможенных пошлин. 

Социальная политика продворянская. В этом смысле закрепощение кресьян будет усиливаться. Правда снижена подать и прощены недоимки. А дворянство становилось монополией с правом иметь землю и крепостных. Узаконена торговля людьми. Ссылка в сибирь неугодных крестьян.

Сенат в 68-м году отменил привелегию покупать крестьян и землю для тех, кто по табелю  о рангах получил дворянство.

Отменила смертную казнь. Сохранилось битьё кнутом. Отправляли на каторгу с пожизненным заключением.

В 1755 году указом Елизаветы Петровны открыт Московский Университет. Тут не было изначально теологического факультета. Были Юридический, Филосовский, Медицинский. Было всего 10 профессоров. 

Главная война: семилетняя. Если бы не смерть Елизаветы, Калининград был бы присоединён раньше. Только это спасло Фридриха Второго от полного краха.

Именно при Елизавете фаворитизм расцветает пышным цветом.

В 1961-м году Елизавета умирает. На престол вступает Пётр Третий, сын Анны Петровны, внук Карла Двенадцатого. В Россию приехал ситой. В детстве не повезло с воспитателями. Елизавета его пыталась готовить к высокому положению. Женится на Софие Фридерика Августа Англьт Цебская (Екатерине Второй). 

Екатерина Вторая будет рисовать крайне неприятный портрет Петра Третьего. 
Секуляризация церковных Земель. Вывод гвардии из столицы. Его начинания перечёркиваются методами, которыми он их проводил. Главной ошибкой Петра была недооценка своей супруги, которая в 1962-м году Пётр был арестован, умер. Косвенно Екатерина причастна к смерти. На просьбы, чтобы его отпустили, она игнорировала.
