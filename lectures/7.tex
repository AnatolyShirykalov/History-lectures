В прошлый раз мы остановились на политической структуре управление Московского царства. Полит культура стала довольно авторитарной. И эта система себя в своё время оправдала. Она позволила Московскому государству состояться.

Современная Россия оказалась почти в границах 17-го века. Как раз Московское государство.

Самодержавие не нужно рассматривать как тиранию. Это тирания, скажем так, потенциально. В нашей истории было не так много тиранов. Ну Иван Грозный хоть и правил долго, он был фигурой неординарноый. Беспорядочные личные отношение, бесконечные жениться.
Почему самодержавие не синоним тирании. 
\begin{enumerate}
\item Общество было традиционным, а значит, религиозным. В Бога все верили, все имели представление о страшном суде. А церковь воспитывала государей так, чтобы он не правил казня и милуя, но как христианин. Надо миссию Земную исполнить.

Небольшой анектод. Отношение Сталина к И. Грозному. Сталин склонен был идеализировать. Вот что сказал Черкасов, что Сталин говорил, что много положительного Иван сделал, оправдывал репрессии. И Сталин говорит: если бы Иван казнил ещё несколько семей, Россия не знала бы смутного времени. А он казнил одну семью и потом полгода грехи замалива "--- вот и не успел.

На западе очень уважают Аристотеля. И он считал тиранию худшей формой правления.

То есть первое: релизиозные издержки и связанные с этим проблемы. У нас сейчас есть законы, но они не соблюдаются. Самое надёжное: вера и страх, это у всех было.
\item Опыт. Опыт жизненный, опыт предков, традиций. Он показывает, что люди "--- это не механизмы. И собака укусить может. Европе бывало, что королей убивали. У нас были случаи убийства императоров. Накапливали опыт: я конечно самодержавный, но нужно считаться; не по закону, а по человечески.

Люди не механизмы, а хомо-сапиенс. Люди все смертные, они все болеют, все устают. Включая первых лиц. Пётр первый говорил: я не святой, не волшебник, не могу творить чудеса, не могу работать по двадцать четыре часа. Управляющий должен делегировать полномочия. А те ещ кому-то делегируют. Николай Первый сказал: не я управляю Россией, а десять тысяч начальников. Бюрократия "--- это не механизм, а люди. Формально всё будем исполнять, а на деле саботаж. В современной России каждый думает и себе.
\end{enumerate}
Типичная фигура Московская "--- это не Иван Грозный. Типичная фигура "--- это Алексей Михайлович (Тишайший). Человек был достаточно энегричный, мог и вспылить. Но в том то и дело, что это государь обладавший властью неограниченной, не посягнул ни на чью-то жизнь, ни на чьё-то имущество.

Иван Грозный не типичный. Он революционер. Он традиции поменял. Опричнина "--- это хирургическая операция.

\subsection{Система гос управления в Московскую эпоху (16--17 вв)}
Во главе был царь. Но не ушло понятие Великий Князь. В термином Царь неясно, насколько высокий титул. Понятно, что выше князя. Когда после окончания смуты при переговорах Шведы отказывались признавать первого из Романовых Царём. Ну разорённая страна, бессильная. Не признаем.

Появляется понятие самодержец и всея Руси. Последнее означает претезнию на Украинские земли.

При царе совет «боярская дума». Исходя из традиций здравого смысла о том, одна голова хорошо, а две "--- лучше. Слишком высока цена ошибки. Царь поступает самодержавно, но он оценивает, считается. Вот почему совет. Боярин "--- не титул, который автоматически передаётся по наследство. Это должность, кооторую нужно получить в том числе за заслуги.

Это тоже царскую влась ограничивало. Трудно идти против течения. Даже царю.

Опричнина "--- попытка вырваться из окружения знати.

Следующее. А какие функции боярской думы? Они универсальны. Боярская дума сливалась с монархом. Если они заседали без монарха, то решения всё равно принимались от имени монарха. Все важные дела в полномочия входили.

Кто входил в боярскую думу: небольшой коллектов (численность росла от 15 до 40). Входяла высшая московская власть. Действовал принцип местническтва, то есть более знатный человек отказывался служить менее знатному. Это обеспечивало то, что все высшие посты занимались знатью. А спор был, кто знатнее. Было две категории знати.
\begin{enumerate}
\item Княжата. Те, кто имеет звание князя. Все наследственные, титул не жалуется. Было несколько линий князей. Рюриковичи "--- потомки киевских князей. Гидеминовичи "--- потомки Гидемина, князя Литовского (хотя Литва позже появилась). Третья линия: татрское и монгольское происхождения: их нужно было как-то назвать, назвали князьями. Кто-то потомки Чингисхана "--- знаменитийший покоритель вселенной. 

\item Московская знать. Когда-то князей в Москве было немного. И тогда сформировалась раннемосковская знать. Не сразу вырабатывались фамилии. Знатных людей называли по имени Отчеству. Многие потом получили титулы, но тогда это было не принято. Но их предки занимали высшие Московские посты изначально.
\end{enumerate}

Ещё один элемент власти: Земский Собор. Первый возник в 1\,549, вошёл в историю, как Собор Примирения. Какие собрания считать земским собором?

Невыбортные элементы: царь и боярская дума в полном составе.

Выборные: от дворянская, то есть воины (аналог европейского рыцарства; полноценные войны: конный, людный и оружный. Пассатские (торговцы и ремесленники): такая была структура: в центре города крепость. Но городские стены расстраивались редко, это дорого. Внутри крепости жила знать, за стенами остальные.

Вот выборные и есть земский собор. 
В средние века не увлекались законодательством. Не было закона и земском собор. Но аналог Киевского вече. Орган прямой демократии. Собирались и соборы и вече довольно редко, когда власть потребует. Власть требует, когда власть сомневается в чём-то, или когда война тяжёлая (не любая), когда власть слабая, когда понимает, что её слова мало, нужно подкрепить словом всей земли.
Но когда жизнь наладилась, всё реже собирались соборы. Потому что не было закона собираться с какой-то периодичностью.

Война: не любая война требовала согласования собора. Налоги: тоже не любые налоги требовали собора. Законы тоже на обязательно через собор. Судебники не утверждались собором. Но есть и сборник, который утверждал собор.

Вот декабристов приговорил суд к четвертованию согласно действующим закон. Николай смягчил всем наказание.

Высказываются не все. Высказывается общее соборное время. Только советовали царю. Как поступить, решал сам царь. Но расхождений собора с царём в общем-то не было.

Взятие Азовской крепости могло означает войну с Турцией. Правительство не смотря на согласие Зесмкого собора решило не воевать.

После смерти Фёдоровича Иоановича земский собор принимает на себя полномочия верховной власти. Проводит выборы царя. Последний из выборных царей: Михаил Романов. После этого практика избрания в общем-то прекратилась.

Когда то, что править будет Пётр, а не его старший брат, решали, это решала какая-то московская знать. Объявили зарём Петра. А дальше мятеж стрелецкий. Дальше решено было иметь два царя, никому не обидно.

Ограном власти ещё были «приказы». Аналог министерств. Считает, что в основном были в кремле на ивановской площади. Отсюда «кричать на всю ивановскую».

Во времени статус царя меняется, роль думы, роль собора меняются. Была ли вообще сословно-представительская монархия? Монарх формально не ограничен, но приходится считаться со всеми. В англии возникается на этот счёт парламент. Работает постоянно. Собрание высших вассалов короля. А затем появляется нижняя палата: среднее дворянство и верхушка городской буржуазии.

Боярская дума "--- нечто похожее на палату лордов. Земкий собор был похож на парламент (английские короли могли долго не собирать парламент), но налоги принимались только через парламент.
И у нас в годы смуты: выборы царя. Приглашение польского царевича Владислава. Он должен был налоги вводить с земским собором. Но это не закрепляется и начинают развиваться элементы абсолютизма.

\section{Внутренняя политика Петра Первого}
\subsection{Общий смысл}
Это правление "--- важный рубеж в нашей истории. Этап начинается имперсиким или Петербургским. Период неоднозначный. Во многом хороший. 17 век "--- это самодержавие с патриархом, боярской думой и земским собором. А после этого возникает модель самодостаточная. Абсолютизм европейский. Монарх опирается на армию и бюрократию. На аппарат, а не живые общественные силы.

Докажем, что при Петре Первом сформировалась абсолютная монархия. При П. патриарх ликвидируется. После смерти патриарха, Пётр приказывается выборы не проводить, а назначает исполняющего обязанности.

Украинцы и Белорусы были более образованы. Почему? К вопросу о конкуренции. Бытовому правослацию, нашей церкви не было с кем конкурировать. А в Польше православные были под гнётом католиков. Всё-таки у католиков было хорошо поставлено образование. В университетах главный предмет "--- богословие. 

Конкуренция заставляла учиться. Образованные священники переходили на русскую службу.

Возникает синодальное управление. Назначается святейший синод, вместо единого патриарха. До 1917 не было патриахра, а был синодальный. Объяснение этой реформы дал выходец из Украины Феофан Прокопович: с одной стороны цекрви нужно быть коллегиальной, папа "--- это ересь; коллегиальное устройство более подходит.

Есть и другая причина. Есть два главных человека: царь и патриарх, это порождает стычки. Нужен один главный.

Земские соборы при Петре угасли и после более не собирались. Боярская дума не закрывалась, но она сама рассосалась. Пётр не жаловал чины. Пётр создаёт сенат. Он удачно или неудачно замещается боярскую думу. В сенат Пётр назначает произвольно "--- не учитывает знатность.

Пётр насаждая абсолютизм пошёл дальше своих европейских коллег. Но в одном или в двух вопросах власть мнарха была ограничена. Первый вопрос: это вера. Становится королями в англии католикам запрещено, но формально мусульманам, например, можно, просто не прописано.

Прописываются чёткие законы, как власть передавать. Есть несколько систем
\begin{enumerate}
\item Скандинавская: старшему ребёнку не зависимо от пола;
\item Кастильская: имеют преимущество мужчины.
\item Австрийская: по мужской линии по старшинству, но если нет мужчин, то по воле боярской.
\item Солическая: женщина не может быть монархом, но регентом можно. Если мужчин нет, династия пресечена. 
\end{enumerate}

Пётр первый постановил, что престол постановил по завещанию. То есть нет закона. Как укажу, так и будет. Формально можно передавать престол человеку не из царской семьи.
