Зовут меня Сергей Владимирович Пронкин, профессор. Буду читать у вас курс истории России до примерно 1991-го года.

В чём проблема? Курс разорван на два года. Не совсем обычно это. А вторая проблема заключается в том, что в этом семестре я буду у вас только лекции вести, а семинары будет вести другой преподаватель. Чтобы контроль над вами не терять, буду, уж извините, смотреть на посещаемость лекций. В этом семестре зачёт, в~следующем "--- экзамен.

Предмет не сложный, но всё-таки это МГУ, тут есть требования и знания истории вы должны продемонстрировать. На экзамене во втором семестре вы, скорее всего, будете отвечать только по материалам второго семестра.

Будет работа с первоисточниками. В конце пойдём по советским эпохам. Биография Ленина и выдержки из его главной работы. Затем Сталин, затем Хрущёв и его выступление, осуждающее сталина. Надо уметь дать портрет политического деятеля. Главное "--- работа с документами, чтобы вы прочитали и сказали, что пришли к~таким-то выводам.

Я последние лет двадцать занимаюсь проблемами гос. управления.

\section{Особенности исторического исследования}
Вопросы.
\begin{enumerate}
\item Отличия иследования истории от точных и естественных наук.
\item Принципы исторического исследования.
\end{enumerate}
Чтобы касается пользы от истории я вам не буду объяснять. Это одна из важнейших наук. Сошлюсь на мнение то ли Маркса, то ли Энгельса
\begin{quote}
Человечество знает только одну науку: историю. Она разделяется на историю людей и историю вещей.
\end{quote}

Кто-то сказал: можно не знать многих вещей, но интеллигентным человеком считаться. А вот не знать историю и считаться культурным даже нельзя. Есть, конечно, анекдот про Шерлока Холмса, "--- это крайний пример.

История нужна для развития аналитических навыков. Аналитические навыки отличны от тех, которые нужны в точных науках. Ну а потом чтобы быть сознательным гражданином, нужно быть немного историком. Иначе будем постоянно на одни и те же грабли наступать.

Мы видим сейчас, что Россия после десятков лет реформ не благоустроена совсем. И это, как выясняется, даже опасно. Это отчасти от незнания собственной истории так получается.

В чём же отличие от исследования в сфере истории от исследований в сфере точных наук.
\begin{enumerate}
\item Историк обычно не наблюдает то, что исследует. Наука о прошлом. А когда прошлое начинается? Мы знаем прошлое и думаем о будущем, а настоящее неуловимо.

Я хоть и историк, а занимаюсь современностью. Можно ли историю писать о событиях, которые вчера закончились? Вот с точки зрения стандарта. Есть точка зрения, что нельзя писать о том, что произошло меньше десятилетие назад.

Вот летописи читаны-перечитаны. Каждое слово. Новое изобрести трудно. А то, что ближе к нам и документов больше: больше шансов стать первопроходцем, имя себе сделать. Почему же рекомендуется заниматься проблематикой, которая отделена десятилетями, столетиями? Страсти утихают, проще оказаться именно на стороне науки, адекватно оценивать. Когда мы изучаем драматическое событие недавнего прошлого, когда ещё действующие лица живые, легко впасть в публицистику.

Вторая причина, почему трудно изучать недавнее. Не вполне ясны долгосрочные последствия. Ведь сегодня кажется одно, а через год уже иначе кажется, а через десять лет "--- совсем непонятно. Можно гадать, но никто не знает, ни Путин, ни Меркель, ни Обама.

Но это создаёт трудности. Трудно разбирать проблему, которую мы не видим. Какова логика исследования исторического? Она, если упростить, состоит из трёх элементом Ф, И, О.
\begin{itemize}
\item[Ф] Факты. Первый элемент "--- это накопление фактов. Причём можно многое знать, но сконструировать не те концепции, уроки не извлечь. Но хуже, когда делают выводы, не зная факты. Возникает вопрос: а сколько фактов? Многие исследователи берутся за работу и в фактах тонут. Очень много времени нужно потратить, чтобы на уровне документов всё изучить.

Давно закончилась эпоха энциклопедистов. Все проблемы довольно узкие. Даже у тех, кто Нобелевские премии получает. Но даже по узкой проблеме нужно изучить огромное количество фактов. Нужно знать меру, соблюдать балас. Это вообще в любой науке.

Ну ведь и не специалиста по математике. На уровне школьного учителя "--- пожалуйста. Можно такого назвать математиком. А специалисты всегда узкого профиля.

Ну вот ВОВ. Очень важная проблема. Сколько документов нужно изучить? Миллионы. Я могу быть специалистом по одной конференции "--- дело важное и компактное, несколько дней. Но опять же: получится мнение России. А мнения других стран? Нужно туда ехать и исследовать их архивы. Нужно ехать на полгода, может даже на год.

История "--- тоскливое занятие. Весь день листал, ничего не нашёл, день пропал. А с другой стороны, а вдруг там именно, что у меня в руках, то, что я ищу. И вот такой соблазн сделать всё побыстрее, а с другой стороны "--- долг историка «я всё пролистал, всё просмотрел».

Итак знание истории "--- это в первую очередь знание фактов. Хотя это и непросто.
\item[И] После знания фактом, мы их \textbf{изучаем}. Есть такое понятие «критика источника». Историк в чём-то оказывается в положении следователя. Всегда нужно понимать, насколько источник достоверен. Дело не в том, что «октябрьская революция была в 1917-м году» "--- этот факт достоверен. Факт мы изучаем через источник. Факт бывает материальный (геологические исследования), бывают письменные источники: официальные записки, частная переписка, мемуары (литературные, политические, военны). Но известно, что мемуары являются одним из самых недостоверных источников, с которым нужно относиться очень осторожно. Это "--- показания свидетеля. Этот свидетел был за какую сторону? Нужно относиться с известным скепсисом. А вот если это был прохожий, ему, наверное, можно доверять. Но и тут не всегда. Может быть национальный момент, или просто плохо видит, шёл без очков.

Итак про мемуары: а он в этом событии участвовал сам или ему дедушка рассказывал, когда ему было три года? В последнем случае в голове уже всё перепуталось. Второй вопрос: а он сам участвовал, воевал, боролся или просто оказался случайным свидетелем? В последнем случае больше доверия. Это даже не сознательно, это просто психология. Если вы пишите мемуары, то вам хочется свои ошибки замять, а достоинства преумножить.

Смотрел телепередачу про Будёнов. Гражданская война. И вот в этой передачи показаны страницы мемуаров, где он пишет истории войны. Смотрю и выясняю, почему все письма написаны одним почерком, одними чернилами? Потому что это не письма, а мемуары. Через 20 лет я пишу и оправдываю себя, а что я тогда думал, уже потеряно. Письма более достоверны. Будёнов стремился показать, что в гибели своих войск не виноват.

Итого, второе есть критика источников. Письмо родным лучше мемуаров, а лучше официальные источники.
\item[О] После этого делаем выводы, или \textbf{Обобщения}.
\end{itemize}
\end{enumerate}
Я пишу Ф--И--О через чёрточки. Они связаны. Источник может исказить факты. И есть такой феномен: «идти за источником». Вы читаете документы про человека, он вам становится как родной, вы его оправдываете, смотрите его глазами. А как историк вы должны быть скептически недоверчивым. Нельзя поддаваться авторитету и обаянию. Вы, когда пишите биографию, работаете на себя, как историка.

Недавно смотрел один кинофильм. Часто люди становятся в произведениях искуства противоположностями самого себя. Вы считали, что Мор хороший, а он плохой. А другой наоборот. Но это литература, а не история.

Итак, мы можем неправильно понять источник и в результате прийти к неправильному выводу. Многие источники наши и не наши о Ленине, Сталина, Брежнева\dots отличаются, даже противоположные. Значит, где-то произошла ошибка, сознательная или несознательная.

Какие ещё различия с точными и естественными науками. Если речь идёт о событии, то мы не можем поставить эксперимент и посмотреть, как события бы развивались, если бы Наполеон себя лучше чувствовал. Может любой малый факт смог бы повернуть сражение в другую сторону. Если ты в истории пришёл к каким-то выводам, то на «а ты докажи», нечего ответить.

История касается сложных процессов. Считается, что процессы по сложности делятся на
\begin{itemize}
\item физические
\item химические
\item биологические
\item социальные
\end{itemize}
в порядке возрастания сложности. Почему социальны самые сложные? Всё очень просто. Вот определяете орбиру планеты, она не меняется, можно долго наблюдать. В биологии животные могут менять своё поведение, но не так как люди. Люди "--- это почти свобода воли. Тут влияет всё, пропаганда, например.

Кто-то сказал «какая честь, когда нечего есть». Так думают не все, но есть и такие, это трудно учитывать. Мы можем определить среднего человека. Есть оболванивание, строятся полит технологии. Это не значит, что отдельный человек поступит, как средний. Мы часто решаем проблему, не зная, почему так поступили. Пойти в кино или в библиотеку? Не плюс или минус, а и то и другое.

Один президент США эпохи великой депрессии.
Дайте мне одноглазово экономиста! А то каждый, у кого ни спросишь, говорит «с одной стороны так, с другой стороны так».

Есть такая наука Эстетика. А в эстетике есть объективная база фактов? У всех гуманитарных наук есть большие проблемы.

Наука "--- это то, что стремится к объективным выводам. Так можно ли, раз всё так сложно, считать историю наукой? Многие литераторы, творцы: Пушкин, Карамзин, были неплохими историками. Многие историки были неплохими писателями.

Можно прийти к выводам объективным, научным. Я напомню, чем древнее событие, тем меньше источников, чем меньше источников, тем больше соблазна для фантазии. Поэтому учёные могут выдвигать смелые оригинальные концепции. Для достижения объективности нужно следовать таким правилам.
\begin{enumerate}
\item Подход диалектический. Вы, наверное, с ним знакомы, я применю его к истории. Это законы диалектики Гегеля. В истории трудно разложить на хороший"---плохой. Можно и то и другое перечислять. Реформа хорошая? Хорошая, но вот это будет плохо. Борьба с коррупцией хорошо? Хорошо, но кто-то садится в тюрьму, ему плохо. Одно и то же являние содержит и хорошее и плохое.

Сергей Юльевич Витте закончил математический факультет Одесского (Новороссийского) университета. Был и министром финансов. Так вот Витте, чьи мемуары считаются очень недостоверными, кстати, вспоминал события 1905-го года. Спросили: «а вот не дают денег на поддержание музея Декабристов». Ответил: «Ну и правильно, они же власть не поддерживали». Простые люди не могут оценивать дела особых. Нечего лезть со свиным рылом. Знаете приказ: патронов не жалеть, холостых залпов не давать. Витте не согласился с критикой в обществе. Он сказал: трудно оценить целовека, нужна не одна фраза, а целый роман. Нет хороших людей, которые не сделали бы ничего плохого, и наоборот нет хороших\ldotst

Диалектика "--- это ещё когда одно и то же являение может в разных контектах может быть плохим и хорошим. Жёсткость "--- это плохо? Обычно да, но иногда помогает спасти сотни, тысячи, миллиона людей. Говорят «истина конкретна».
\end{enumerate}
